\documentclass[12pt, a4paper]{report}
\usepackage{epsfig}
\usepackage{subfigure}
%\usepackage{amscd}
\usepackage{amssymb}
\usepackage{amsbsy}
\usepackage{amsthm}
%\usepackage[dvips]{graphicx}
%\usepackage{natbib}
%\bibliographystyle{chicago}
%\usepackage{vmargin}
% left top textwidth textheight headheight
% headsep footheight footskip
%\setmargins{3.0cm}{2.5cm}{15.5 cm}{22cm}{0.5cm}{0cm}{1cm}{1cm}
%\renewcommand{\baselinestretch}{1.5}
%\pagenumbering{arabic}
%\theoremstyle{plain}
%\newtheorem{theorem}{Theorem}[section]
%\newtheorem{corollary}[theorem]{Corollary}
%\newtheorem{ill}[theorem]{Example}
%\newtheorem{lemma}[theorem]{Lemma}
%\newtheorem{proposition}[theorem]{Proposition}
%\newtheorem{conjecture}[theorem]{Conjecture}
%\newtheorem{axiom}{Axiom}
%\theoremstyle{definition}
%\newtheorem{definition}{Definition}[section]
%\newtheorem{notation}{Notation}
%\theoremstyle{remark}
%\newtheorem{remark}{Remark}[section]
%\newtheorem{example}{Example}[section]
%\renewcommand{\thenotation}{}
%\renewcommand{\thetable}{\thesection.\arabic{table}}
%\renewcommand{\thefigure}{\thesection.\arabic{figure}}
%\title{Research notes: linear mixed effects models}
%\author{ } \date{ }


\begin{document}
\author{Kevin O'Brien}
\title{Spring 2011}

\addcontentsline{toc}{section}{Bibliography}
\newpage
\section*{Question 6}
{\Large
\[ A = \left(
\begin{array}{ccc}
1 & 1 & 0 \\
0 & 0 & 1 \\
0 & -2 & -3
\end{array} \right) \]
\begin{itemize}
\item Last Class - We found the Eigenvalues of A
\item Characteristic Equation $(\lambda - 1)(\lambda + 1)(\lambda + 2)=0$
\item Eigenvalues $\lambda = \{-1,-2,1\}$
\item To find Eigenspaces
\item (Tutorial 6 Question 5 is a good example for this question)
\end{itemize}


\[ A = \left(
\begin{array}{ccc}
1 & -1/3 & 0 \\
0 & 2/3 & 1 \\
0 & -2/3 & -3
\end{array} \right) \]

\newpage
\begin{itemize}
\item Find the Eigenspaces :
 Solve $(\lambda I -A)e=0$
\item $(\lambda I -A)$ is computed below. First $\lambda =-2$

\[
\left(
\begin{array}{ccc}
-2 & 0  & 0  \\
 0 & -2  & 0  \\
 0 &  0 & -2
\end{array}
\right) -
\left(
\begin{array}{ccc}
1 & 1  & 0  \\
 0 & 0  & 1  \\
 0 &  -2 & -3
\end{array}
\right) = \left(\begin{array}{ccc}
-3 & -1 & 0 \\
0 & -2 & -1 \\
0 & 2 & 1
\end{array} \right)\]
\end{itemize}

\newpage
{\Large
\begin{itemize}
\item The Eigenspace weightings are the solutions to the following. (you can let $e_1$ = 1, then re-weight if necessary)

\[\left(\begin{array}{ccc}
-3 & -1 & 0 \\
0 & -2 & -1 \\
0 & 2 & 1
\end{array} \right)
\left(\begin{array}{c}
e_1 \\
e_2 \\
e_3
\end{array} \right) = \left(\begin{array}{c}
0 \\
0 \\
0
\end{array} \right)\]
\item The solution is $e_1=1$, $e_2=-3$ $e_3=6$

\item For $\lambda =-1$ and $\lambda =1$. The matrices are

\[\left(\begin{array}{ccc}
-3 & -1 & 0 \\
0 & -2 & -1 \\
0 & 2 & 1
\end{array} \right) \left(\begin{array}{c}
e_1 \\
e_2 \\
e_3
\end{array} \right) = \left(\begin{array}{c}
0 \\
0 \\
0
\end{array} \right)\]

\[
\left(\begin{array}{ccc}
-3 & -1 & 0 \\
0 & -2 & -1 \\
0 & 2 & 1
\end{array} \right)\left(\begin{array}{c}
e_1 \\
e_2 \\
e_3
\end{array} \right) = \left(\begin{array}{c}
0 \\
0 \\
0
\end{array} \right)\]

\item The P matrix is therefore
\[
\left(\begin{array}{ccc}
1 & -1 & 1 \\
-3 & 2 & 0 \\
6 & -2 & 0
\end{array} \right) \]
\end{itemize}
}




\newpage
%----------------------------------- %
{\Large
\begin{itemize}

\item Normalising the columns
\[
\left(\begin{array}{ccc}
1/ \sqrt{46} & -1/\sqrt{9} & 1/\sqrt{1} \\
-3/ \sqrt{46} & 2/\sqrt{9} & 0/\sqrt{1} \\
6/ \sqrt{46} & -2/\sqrt{9} & 0/\sqrt{1} \\
\end{array} \right)=
\left(\begin{array}{ccc}
0.147 & -0.333 & 1 \\
-0.442 & 0.666 & 0 \\
0.885 & -0.666 & 0
\end{array} \right) \]

\item Normalization : divide each element by magnitude of column vector ($ \sqrt{1^2 +(-3)^2 +6^2} =\sqrt{46}$)

\end{itemize}
}

\newpage
{\Large

\[ \left(
  \begin{array}{c}
    \dot{x}_1 \\
    \dot{x}_2 \\
    \dot{x}_3 \\
  \end{array}
\right)\left(
\begin{array}{ccc}
1 &1 &0\\
0 &0& 1\\
0 &-2& -3\\
 \end{array}
       \right)
        \left( \begin{array}{c}
    x_1 \\
    x_2 \\
    x_3 \\
  \end{array}
\right)\]

Boundary conditions

\[ X(0) = \left( \begin{array}{c}
    1 \\
    1 \\
    0 \\
  \end{array}
\right)\]

Linear Transformation
\[ \left(
  \begin{array}{c}
    x_1 \\
    x_2 \\
    x_3 \\
  \end{array}
\right)\left(
\begin{array}{ccc}
\left(\begin{array}{ccc}
1/ \sqrt{46} & -1/3 & 1 \\
-3/ \sqrt{46} & 2/3 & 0 \\
6/ \sqrt{46} & -2/3 & 0 \\
\end{array} \right)
 \end{array}
       \right)
        \left( \begin{array}{c}
    \hat{x}_1 \\
    \hat{x}_2 \\
    \hat{x}_3 \\
  \end{array}
\right)\]


}
\newpage

{\Large
\begin{itemize}
\item Let $\|A\|$ represent the norm of A.
\item The condition number is defined as $\|A\| \times \|A^{-1}\| $ 
\item Condition number is the measure of how well  conditioned a matrix is. The smaller the value of $\kappa(A)$, the more accurate the solution of Ax=b
\item Inverse of A found yesterday, using Elementary Row Operations.
\item Can quickly compute the inverse of L and U by the same method.
\item Recall $A^{-1} = (LU)^{-1} = U^{-1}L^{-1}$
\end{itemize}


\[A = \left(\begin{array}{ccc}
1&-3&1\\
2&-5&4\\
2&-2&11\\
\end{array}\right) \qquad A^{-1}=\left(\begin{array}{ccc}
-47	&	31	&	-7	\\
-14	&	9	&	-2	\\
6	&	-4	&	1	\\
\end{array}\right)\]

\begin{itemize}
\item $\|A\| = \mbox{Max} \{(1+|-3|+1, 2+|-5|+4, 2+|-2|+11  \} = \mbox{max} \{5,11,15 \}  = 15$
\item $\|A^-1\| = \mbox{Max} \{(|-47|+ 31 + |-7|, |-14|+ 9 +|-2|, 6+|-4|+1  \} = \mbox{max} \{85,25,11 \} $
\item $\kappa(A) = 15 \times 85 = 1275$
\end{itemize}
}
%----------------------------------------------------%

{\Large
\section*{Question 4}
\[a_o + a_1x + a_2x^2 = \alpha_1(1+2x) + \alpha_2(2x+x^2)+\alpha_3(4+2x-3X^2)\]

\[=(\alpha_1+4\alpha_3) + (2\alpha_1 + 2\alpha_2 + 2\alpha_3)x + (\alpha_2-3\alpha_3)x^2\]

\begin{itemize}
\item $a_o =  \alpha_1+4\alpha_3$
\item $a_1 = 2\alpha_1 + 2\alpha_2 + 2\alpha_3$
\item $a_2 = \alpha_2-3\alpha_3$
\end{itemize}

\[\left(\begin{array}{ccc}
 a_0\\
 a_1\\
 a_2\\
\end{array}\right)= \left(\begin{array}{ccc}
1 &0& 4\\
2 &2& 2\\
0 &1& -3\\
\end{array}\right)
\left( \begin{array}{ccc}
 \alpha_1\\
 \alpha_2\\
 \alpha_3\\
\end{array}\right)\]

\[
\left|\begin{array}{ccc}
1 &0& 4\\
2 &2& 2\\
0 &1& -3\\
\end{array}\right| =0? \mbox{Yes}\]
}











\end{document}
			











%-----------------------------------------------------------------------------------------------------%
\newpage

\subsection{Bendix Carstensen's data sets}
\citet{bxc2008}describes the sampling method when discussing of a motivating example.Diabetes patients attending an outpatient clinic in Denmark have their $HbA_{1c}$ levels routinely measured at every visit.Venous and Capillary blood samples were obtained from all patients appearing at the clinic over two days.

Samples were measured on four consecutive days on each machines, hence there are five analysis days.Carstensen notes that every machine was calibrated every day to  the manufacturers guidelines.





%-----------------------------------------------------------------------------------------------------%
\bibliography{transferbib}
\end{document} 