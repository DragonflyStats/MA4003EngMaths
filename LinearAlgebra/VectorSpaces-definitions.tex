\subsubsection{A Linearly Indepedendent Set }

The Set $\{v_1, v_2, \ldots, v_k\}$ is a linearly independent set of vectors if


\[ \alpha_1 v_1 + \alpha_2v_2 + \ldots + \alpha_k v_k = 0 v \]

This implies $\alpha_i = 0 for each $i$ \]


\subsubsection{A Spanning Set for \textbf{V}}
A spanning set for \textbf{V} is a set of vectors $\{v_1, v_2, \ldots, v_k\}$  in \textbf{V} is a 
spanning set for \textbf{V} if each $ v \in \textbf{V} $ is expressible as a linear 
combination of $\{v_1, v_2, \ldots, v_k\}$.


\subsubsection{A basis for \textbf{V}}

A set  $\{v_1, v_2, \ldots, v_k\}$ in \textbf{V} is called a basis of \textbf{V} if it is

\begin{itemize}
\itme[i] a linearly independent set
\item[ii] a spanning set for \textbf{V}
\end{itemize}

%=================================================%
\subsection{Basis Theorem}

Let \textbf{V} be a finite dimensional vector space. Then any two bases of \textbf{V} have 
the same number of elements.

The dimension of a finite dimensional vector space \textbf{V} is the unique number of elements
in a basis of \textbf{V}.




%================================================%



Example 2 

%================================================%

\[(3,4,5) = \alpha_1(2,1,0) + \alpha_2(-1,-1,-1) \]

\begin{itemize}
\item $3 = 2 \alpha_1 - \alpha_2$
\item $4 = \alpha_1 - \alpha_2$
\item $5 = -\alpha_2$
\end{itemize}

Necessarily $\alpha_1 = -1$ and $\alpha_2 = -5$

\[(3,4,5) = 1(2,1,0) -  5(-1,-1,-1) \]
\[(0,1,2) = -1(2,1,0) - 2(-1,-1,-1) \]


Each vector is \textbf{S} is a lienar combination for vectors in \textbf{T}.
\[ \mbox{span( S \cup T) = \mbox{span}(T)\]
