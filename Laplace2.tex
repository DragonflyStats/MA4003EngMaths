



sin(3t) = sin(3(t-2) + 6)

sin(A + B) = sin (A)Cos(B) + cos(A)Sin(B) 

sin(3(t-2) + 6) = sin(3(t-2))cos(6) + cos(3(t-1))sin(6)

sin(3) = 0	           cos(6) = 1

sin(3(t-2) + 6) = sin(3(t-2))

f(t) = sin (3t) +u2(t)sin (3(t-2))

F(s) =(3)2s2+ (3)2-e-2s(3)2s2+ (3)2

F(s) =(1-e-2s)92s2+ 92
 
 
Autumn 0607 Question 1aiii : Periodic function
 
Period of function is 2
 
Base function (b (t))
 
Base function repeated infinitelyf(t) = f(t+p)

Expression can be re-arranged as follows:
 
b(t) =u0(t) +u1(t) -2u2(t) 
 
Laplace transform of base function
 
B(s) =1s+e-ss-2e-2ss 
 
 
Required Laplace transform is derived from base function as follows
 
 
G(s) =11-e-psB(s) 
 
G(s) =11-e-2s1+e-s-2e-2ss 
 
 
 
Let     x =e-s
 
        x2=(e-s)2=e-2s
 
G(s) =11-x21+ x -2x2s
 
G(s) =1x2-12x2-x-1s
 
2x2-x-1 = (2x-1)(x+1)
x2-1 = (x-1)(x+1)
 
 
G(s) =1x-12x-1s 
 
G(s) =2e-s-1s(1-es)
 
  
Week 8

Inverse Laplace transforms

Q1 - a - i

Q1 - a - ii

Q1 - a - iii

Inverse Laplace transforms

Q1 - b - i

Q1 - b - ii

Q1 - b - iii
Inverse Laplace transforms


Q2 - a
Solving Differential equations using Laplace transforms

Q2 - b 
Solving Integral equations using Laplace transforms


base function B(s)

Base function repeated infinitelyf(t) = f(t+p)

11-e-ps

G(s)  = \frac{B(s)}{}

 
Part 4 Solving ODEs using Laplace Transforms 
 
d2ydt2+dydt+ y = 1 + t 
 
Alternative notation
 
d2ydt2y(t)				dydty(t)

Initial Conditions are often (but not always) zero	y(0) = 0y(0) = 0
 
 
y+y+ y = 1 + t
 
L[y] =s2Y(s) - sy(0)-y(0)            note the initial conditions. L[y] =s2Y(s)
 
L[y] = sY(s) - y(0)                           note the initial conditions. L[y] = sY(s) 
 
L[ y] = Y(s)
 
Ld2ydt2+dydt+ y=(s2+s+1)Y(s)	Quadratic Term multiplied by Y(s)


 


f(t) =u0(t) -u1(t)[sin (3t)] = sin (3t) -u1(t)sin (3t)

sin(3t) = sin(3(t-1) + 3)

sin(A + B) = sin (A)Cos(B) + cos(A)Sin(B) 

sin(3(t-1) + 3) = sin(3(t-1))cos(3) + cos(3(t-1))sin(3)

sin(3) = 0	           cos(3) = -1

sin(3(t-1) + 3) = -sin(3(t-1))

f(t) = sin (3t) +u1(t)sin (3(t-1))

F(s) =(3)2s2+ (3)2+e-s(3)2s2+ (3)2

F(s) =(1+e-s)92s2+ 92


End of Year Exam  2009 Question 1bi

Find the inverse laplace transform of G(s)

G(s) =2s-4s2- s - 6

s2- s - 6 = (s-3)(s+2)

G(s) =2s(s-2)(s+3)-4(s-2)(s+3)

entry no 12 and 13 with a =3 and b =  -2


End of Year Exam  2009 Question 1b(ii)
Find the inverse laplace transform of G(s)

G(s) =e-4ss2- 3s


G(s)in forme-asF(s)witha = 4.


F(s) =1(s)(s-3)


f(t) =13[e3t- 1 ]

End of Year Exam  2008 Question 1b(iii)

Find the inverse laplace transform of G(s)

G(s) =tan-1(s)

tan-1(s) ds


Example 1
Find g(t), the inverse Laplace transform of G(s) =2s3
Consider this in form G(s) = -F(s)
Necessarily 
 
Finding the inverse Laplace transform of F(s)
 
f(t) = t
 

g(t) = tf(t) =t2
 
Summary:
∙       Given G(s), we consider it in form G(s) =-F(s)
∙       We negate G(s) and integrate it to find F(s).
∙       We get the inverse laplace transform of F(s) to find f(t).
∙       We multiply f(t) by ‘t’ to find g(t)
 
Example 2
Find the inverse Laplace transform of lnss-1
 
To solve this we use a same approach to the one in the previous example, but in reverse.
 
∙       We consider the given equation in form F(s).
∙       We negate F(s) and differentiate it to find G(s).
∙       We get the inverse Laplace transform of G(s) , yielding g(t).
∙       We divide g(t) by ‘t’ to find f(t)
 
 F(s) = lnss-1= ln(s) - ln(s-1)      	 
G(s) =-F(s) = -dds( ln(s)) +dds(ln(s-1))

G(s) =dds( ln(s-1)) -dds(ln(s))

G(s) =1s-1-1s

g(t) =et- 1

f(t) =g(t)t=et- 1t


 
G(s) = lns2s2-1

 lns2s2-1=ln(s2) -ln(s2-1)

ln(s2) = 2ln(s)


dds2ln(s) =2s


To compute  ddsln(s2- 1), apply the chain rule.


ddsln(s2- 1) = 2s1s2-1 

 F(s) =2ss2-1-2s


 g(t) = 2 cos(t) -2

f(t) =g(t)t=2 cos(t) -2t

Laplace Question 1bii
 
 
k(s-a)(s-b)=ka-bs-a+-ka-bs-b
 
 
s(s-a)(s-b)=a+ka-bs-a+-b-ka-bs-b


 
Example 4 	 	         Find the inverse Laplace transform of  F(s)

F(s) =tan-1(s)

G(s) =-F(s) = -ddstan-1(s)

G(s) =-1s2-1

g(t) = -sin(t)

f(t) =g(t)t=-sin(t)t
 
 
Using Laplace transforms to solve integral equations#
 
 y(t) = 1+t+0tcos(t-u).y(u)du
 
L[y(t)] = Y(s)
 
L[1+t] =1s+1s2=s+1s2
 
L[0tcos(t-u).y(u)du] =L[cos(t)]L[y(t)] =s Y(s)s2+1
 
Y(s) =s+1s2+s Y(s)s2+1
 
(s2+1) Y(s)s2+1-s Y(s)s2+1=s+1s2
 
(s2-s +1) Y(s)s2+1=s+1s2
 
 

y(t) = cos(t)+ 20tsin(t-u).y(u)du

 
 
L[y(t)] = Y(s)
 
L[cos(t)] =ss2-1
 
L[0tcos(t-u).y(u)du] =L[cos(t)]L[y(t)] =s Y(s)s2+1

