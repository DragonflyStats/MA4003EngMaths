\documentclass[]{article}

%opening
\title{MA4003 Scratch Pad}
\author{}

\begin{document}

\maketitle

%------------------------------------------------------ %
\section{Laplace Transforms}

\[\mathcal{L}\left\{f(t)\right\} = F(s)\]
\begin{itemize}
\item $f(t)$ : function in time domain
\item $F(s)$ : function in frequency domain
\end{itemize}
The Laplace transform of a function $f(t)$, defined for all real numbers $t \geq 0$, is the function $F(s)$, defined by:
\[F(s) = \mathcal{L} \left\{f(t)\right\}(s)=\int_0^{\infty} e^{-st} f(t) \,dt. \]
%------------------------------------------------------ %
\subsection{Inverse Laplace Transform}
\[
f(t) = \mathcal{L}^{-1} \{F(s)\} \]

%------------------------------------------------------ %
\subsection{Frequency differentiation}	 
\[t f(t)  \rightarrow 	 -F'(s) \] 	
F′ is the first derivative of F.

\subsection{Heaviside step function}	 
The Heaviside step function, or the unit step function, usually denoted by $u$), is a discontinuous function whose value is zero for negative argument and one for positive argument. 

\subsection{Partial Fraction Expansion}
The impulse response is simply the inverse Laplace transform of this transfer function:
\[h(t) = \mathcal{L}^{-1}\{H(s)\}.\]
To evaluate this inverse transform, we begin by expanding H(s) using the method of \textbf{partial fraction expansion}:

\[\frac{1}{(s+\alpha)(s+\beta)} = { P \over s+\alpha } + { R  \over s+\beta }.\]


\end{document}
